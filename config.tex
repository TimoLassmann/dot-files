% Created 2020-02-27 Thu 18:47
% Intended LaTeX compiler: pdflatex
\documentclass[12pt]{article}
\usepackage{\string~"/.emacs.d/latex_templates/report"}
\author{Timo Lassmann}
\date{\today}
\title{Emacs Configuration File}
\hypersetup{
 pdfauthor={Timo Lassmann},
 pdftitle={Emacs Configuration File},
 pdfkeywords={},
 pdfsubject={},
 pdfcreator={Emacs 26.3 (Org mode 9.1.9)}, 
 pdflang={English}}
\begin{document}

\maketitle

\section{To speed things up}
\label{sec:org9fcba23}

\begin{minted}[frame=lines,linenos=true]{common-lisp}
(setq gc-cons-threshold 16777216)

(setq gc-cons-percentage 0.6)

(setq  message-log-max 16384)

\end{minted}




\section{Package Manager}
\label{sec:orgf528126}
To start we need the package ``package'' 

\begin{minted}[frame=lines,linenos=true]{common-lisp}
(require 'package)

(setq gnutls-algorithm-priority "NORMAL:-VERS-TLS1.3")

(setq package-archives '(("org" . "https://orgmode.org/elpa/")
                         ("gnu" . "https://elpa.gnu.org/packages/")
                         ("melpa"     . "https://melpa.org/packages/")))
(package-initialize)
\end{minted}

Next let's make sure we have use-package

\begin{minted}[frame=lines,linenos=true]{common-lisp}
(unless (package-installed-p 'use-package)
  (package-refresh-contents)
  (package-install 'use-package))
\end{minted}

\section{General Settings}
\label{sec:org74cf5b0}
\subsection{Init File Support}
\label{sec:orgf63d0a3}

Load up a collection of enhancements to Emacs Lisp, including \href{https://github.com/magnars/dash.el}{dash},
\href{https://github.com/magnars/s.el}{s} for string manipulation, and \href{https://github.com/rejeep/f.el}{f} for file manipulation.

\begin{minted}[frame=lines,linenos=true]{common-lisp}
(require 'cl)

(use-package dash
  :ensure t
  :config (eval-after-load "dash" '(dash-enable-font-lock)))

(use-package s
  :ensure t)

(use-package f
  :ensure t)
\end{minted}

\subsection{Trace for obsolete packages}
\label{sec:orgb4f5e35}

\begin{minted}[frame=lines,linenos=true]{common-lisp}
(defun debug-on-load-obsolete (filename)
  (when (equal (car (last (split-string filename "[/\\]") 2))
               "obsolete")
    (debug)))
(add-to-list 'after-load-functions #'debug-on-load-obsolete)

\end{minted}
\subsection{Whoami}
\label{sec:org5c5cd99}

\begin{minted}[frame=lines,linenos=true]{common-lisp}
(setq user-full-name "Timo Lassmann"
      user-mail-address "timo.lassmann@telethonkids.org.au")
\end{minted}

\subsection{Emacs directory}
\label{sec:orgeb5a13f}

\begin{minted}[frame=lines,linenos=true]{common-lisp}
(defconst tl/emacs-directory (concat (getenv "HOME") "/.emacs.d/"))
(defun tl/emacs-subdirectory (d) (expand-file-name d tl/emacs-directory))
\end{minted}

\begin{minted}[frame=lines,linenos=true]{common-lisp}
(message "%s" (tl/emacs-subdirectory "elisp"))
(add-to-list 'load-path (tl/emacs-subdirectory "elisp"))
\end{minted}

\subsection{Basic looks}
\label{sec:org871bf07}
\subsubsection{remove startup screen}
\label{sec:org4307ef2}
\begin{minted}[frame=lines,linenos=true]{common-lisp}
(setq inhibit-startup-message t) 
\end{minted}

\subsubsection{remove bars ets}
\label{sec:orgd941f9d}
\begin{minted}[frame=lines,linenos=true]{common-lisp}
(tool-bar-mode -1)
(menu-bar-mode -1)
(scroll-bar-mode -1)
(mouse-wheel-mode -1)
\end{minted}
\subsubsection{Disable bell}
\label{sec:org292a529}
\begin{minted}[frame=lines,linenos=true]{common-lisp}
(setq ring-bell-function 'ignore)
\end{minted}

\subsubsection{UTF-8}
\label{sec:org92c3a4e}

\begin{minted}[frame=lines,linenos=true]{common-lisp}
(setq locale-coding-system 'utf-8)
(set-terminal-coding-system 'utf-8)
(set-keyboard-coding-system 'utf-8)
(set-selection-coding-system 'utf-8)
(prefer-coding-system 'utf-8)
\end{minted}

\subsection{Turn off arrow keys}
\label{sec:org5cc6b98}

\begin{minted}[frame=lines,linenos=true]{common-lisp}

(require 'no-easy-keys)
(no-easy-keys 1)

\end{minted}
\subsection{Turn off  mouse}
\label{sec:org7d180a0}

\begin{minted}[frame=lines,linenos=true]{common-lisp}
(use-package disable-mouse
  :ensure t 
  :demand t)

(global-disable-mouse-mode)

\end{minted}
\subsection{Diminish}
\label{sec:orga93e0fc}

\begin{minted}[frame=lines,linenos=true]{common-lisp}
(use-package diminish 
  :ensure t 
  :demand t)
\end{minted}

\subsection{Turn off auto-fill mode}
\label{sec:org0cab21e}

\begin{minted}[frame=lines,linenos=true]{common-lisp}
(setq auto-fill-mode -1)
(setq-default fill-column 99999)
(setq fill-column 99999)
\end{minted}

\subsection{Undo / redo}
\label{sec:org0e5cdf8}
According to this article, I get better functionality than the redo+ plugin (which I can’t seem to get working well).
\begin{minted}[frame=lines,linenos=true]{common-lisp}
(use-package undo-tree
  :ensure t
  :diminish
  :init
  (global-undo-tree-mode 1)
  :config
  (defalias 'redo 'undo-tree-redo)
  :bind (("C-z" . undo)     ; Zap to character isn't helpful
         ("C-S-z" . redo)))

\end{minted}

\subsection{Kill this buffer}
\label{sec:org1db48ed}
Assume that I always want to kill the current buffer when hitting C-x k.
\begin{minted}[frame=lines,linenos=true]{common-lisp}
(defun tl/kill-current-buffer ()
  "Kill the current buffer without prompting."
  (interactive)
  (kill-buffer (current-buffer)))
(global-set-key (kbd "C-x k") 'tl/kill-current-buffer)
\end{minted}

\subsection{Tabs}
\label{sec:org52cfe1d}
Never use tabs. Tabs are the devil’s whitespace.

\begin{minted}[frame=lines,linenos=true]{common-lisp}
(setq-default indent-tabs-mode nil)
(setq tab-width 4)
;; (setq-default tab-always-indent 'complete)
\end{minted}

\subsection{Location of mactex (if we are using mac - unlikely\ldots{}}
\label{sec:org30b1831}
Tell emacs about the mactex installation\ldots{}

\begin{minted}[frame=lines,linenos=true]{common-lisp}
(setenv "PATH" (concat "/Library/TeX/texbin" ":" (getenv "PATH")))
\end{minted}

This should do it\ldots{}

\subsection{Dired settings}
\label{sec:orgd442a06}

Taken from: \url{https://github.com/munen/emacs.d}


Ability to use \texttt{a} to visit a new directory or file in dired instead of using \texttt{RET}.
\texttt{RET} works just fine, but it will create a new buffer for every interaction
whereas a reuses the current buffer.

\begin{minted}[frame=lines,linenos=true]{common-lisp}
(put 'dired-find-alternate-file 'disabled nil)
(setq-default dired-listing-switches "-alh")
\end{minted}

\subsection{Path}
\label{sec:org5e0225d}

\begin{minted}[frame=lines,linenos=true]{common-lisp}
;;(let ((path-from-shell (shell-command-to-string "/bin/bash -l -c 'echo $PATH'")))
;;  (setenv "PATH" path-from-shell)
;;  (setq exec-path (split-string path-from-shell path-separator)))

\end{minted}

\begin{minted}[frame=lines,linenos=true]{common-lisp}
(setq temporary-file-directory "/tmp")
\end{minted}

\subsection{Modernizing Emacs}
\label{sec:org2077e97}

Found \href{https://github.com/wasamasa/dotemacs/blob/master/init.org\#init}{here} how to remove the warnings from the GnuTLS library when
using HTTPS\ldots{} increase the minimum prime bits size:
\begin{minted}[frame=lines,linenos=true]{common-lisp}
(setq gnutls-min-prime-bits 4096)
\end{minted}

\subsection{Turn off sleep mode}
\label{sec:org2d0ef56}
I keep hitting this by accidental
\begin{minted}[frame=lines,linenos=true]{common-lisp}
(global-unset-key (kbd "C-z"))
\end{minted}

\section{Basic Functionality}
\label{sec:org47346f0}
\subsection{Highligh current line}
\label{sec:org1f184d5}

hl-line is awesome! It’s not very awesome in the terminal version of emacs though, so we don’t use that. Besides, it’s only used for programming.
\begin{minted}[frame=lines,linenos=true]{common-lisp}
(when window-system (add-hook 'prog-mode-hook 'hl-line-mode))
\end{minted}

\subsection{yes-no to y-n}
\label{sec:orgadf45a2}
\begin{minted}[frame=lines,linenos=true]{common-lisp}
(defalias 'yes-or-no-p 'y-or-n-p)
\end{minted}

\subsection{Async}
\label{sec:orgdcd6436}

Lets us use asynchronous processes wherever possible, pretty useful.
\begin{minted}[frame=lines,linenos=true]{common-lisp}
(use-package async
  :ensure t
  :init (dired-async-mode 1))
\end{minted}

\subsection{Projectile}
\label{sec:org9caf993}

Projectile is an awesome project manager, mostly because it recognizes directories with a .git directory as projects and helps you manage them accordingly.
Enable projectile globally

This makes sure that everything can be a project.
\begin{minted}[frame=lines,linenos=true]{common-lisp}
(use-package projectile
  :ensure t
  :diminish
  :init
  (projectile-mode 1))
\end{minted}

Let projectile call make

\begin{minted}[frame=lines,linenos=true]{common-lisp}
(global-set-key (kbd "<f5>") 'projectile-compile-project)
\end{minted}

\subsection{Insert date}
\label{sec:org496599b}
This is a piece of code from JorgenSchaefersEmacsConfig.
\begin{minted}[frame=lines,linenos=true]{common-lisp}
(defun insert-date (prefix)
  "Insert the current date. With prefix-argument, use ISO format. With
   two prefix arguments, write out the day and month name."
  (interactive "P")
  (let ((format (cond
                 ((not prefix) "%d.%m.%Y")
                 ((equal prefix '(4)) "%Y-%m-%d")
                 ((equal prefix '(16)) "%A, %d. %B %Y")))
        (system-time-locale "en_US.UTF-8"))
    (insert (format-time-string format))))

(global-set-key (kbd "C-c d") 'insert-date)
\end{minted}

\section{Improvements}
\label{sec:org32bafad}
\subsection{Better beginning of line}
\label{sec:orga32abca}

\begin{minted}[frame=lines,linenos=true]{common-lisp}
(defun smarter-move-beginning-of-line (arg)
  "Move point back to indentation of beginning of line.

Move point to the first non-whitespace character on this line.
If point is already there, move to the beginning of the line.
Effectively toggle between the first non-whitespace character and
the beginning of the line.

If ARG is not nil or 1, move forward ARG - 1 lines first.  If
point reaches the beginning or end of the buffer, stop there."
  (interactive "^p")
  (setq arg (or arg 1))

  ;; Move lines first
  (when (/= arg 1)
    (let ((line-move-visual nil))
      (forward-line (1- arg))))

  (let ((orig-point (point)))
    (back-to-indentation)
    (when (= orig-point (point))
      (move-beginning-of-line 1))))

;; remap C-a to `smarter-move-beginning-of-line'
(global-set-key [remap move-beginning-of-line] 'smarter-move-beginning-of-line)
(global-set-key [remap org-beginning-of-line]  'smarter-move-beginning-of-line)
\end{minted}

\section{Terminal}
\label{sec:org03d0dbd}

I have used urxvt for years, and I miss it sometimes, but ansi-term is enough for most of my tasks.
\subsection{Default shell should be zsh}
\label{sec:org3aa3aeb}

I don’t know why this is a thing, but asking me what shell to launch every single time I open a terminal makes me want to slap babies, this gets rid of it. This goes without saying but you can replace bash with your shell of choice.
\begin{minted}[frame=lines,linenos=true]{common-lisp}

(defvar my-term-shell "/usr/bin/zsh")
(defadvice ansi-term (before force-bash)
  (interactive (list my-term-shell)))
(ad-activate 'ansi-term)
\end{minted}

\subsection{Easy to remember keybinding}
\label{sec:org14fe832}

In loving memory of bspwm, Super + Enter opens a new terminal, old habits die hard.
\begin{minted}[frame=lines,linenos=true]{common-lisp}
(global-set-key (kbd "<s-return>") 'eshell)
\end{minted}

\section{Moving around}
\label{sec:orge861b71}

One of the most important things about a text editor is how efficient
you manage to be when using it, how much time do basic tasks take you
and so on and so forth. One of those tasks is moving around files and
buffers, whatever you may use emacs for you will be jumping around
buffers like it’s serious businexss, the following set of enhancements
aims to make it easier.

As a great emacs user once said:

Do me the favor, do me the biggest favor, matter of fact do
yourself the biggest favor and integrate those into your workflow.


\subsection{scrolling and why does the screen move}
\label{sec:org7db6e7e}

I don’t know to be honest, but this little bit of code makes scrolling with emacs a lot nicer.
\begin{minted}[frame=lines,linenos=true]{common-lisp}
(setq scroll-conservatively 100)
\end{minted}

\subsection{which-key and why I love emacs}
\label{sec:org289f323}

In order to use emacs, you don’t need to know how to use emacs. It’s self documenting, and coupled with this insanely useful package, it’s even easier. In short, after you start the input of a command and stop, pondering what key must follow, it will automatically open a non-intrusive buffer at the bottom of the screen offering you suggestions for completing the command, that’s it, nothing else.

It’s beautiful
\begin{minted}[frame=lines,linenos=true]{common-lisp}
(use-package which-key
  :ensure t
  :diminish which-key-mode
  :config
  (which-key-mode))
\end{minted}

\subsection{windows,panes and why I hate other-window}
\label{sec:orgd068791}

Some of us have large displays, others have tiny netbook screens, but regardless of your hardware you probably use more than 2 panes/windows at times, cycling through all of them with C-c o is annoying to say the least, it’s a lot of keystrokes and takes time, time you could spend doing something more productive.
switch-window

This magnificent package takes care of this issue. It’s unnoticeable if you have <3 panes open, but with 3 or more, upon pressing C-x o you will notice how your buffers turn a solid color and each buffer is asigned a letter (the list below shows the letters, you can modify them to suit your liking), upon pressing a letter asigned to a window, your will be taken to said window, easy to remember, quick to use and most importantly, it annihilates a big issue I had with emacs. An alternative is ace-window, however by default it also changes the behaviour of C-x o even if only 2 windows are open, this is bad, it also works less well with exwm for some reason.
\begin{minted}[frame=lines,linenos=true]{common-lisp}
(use-package switch-window
  :ensure t
  :config
  (setq switch-window-input-style 'minibuffer)
  (setq switch-window-increase 4)
  (setq switch-window-threshold 2)
  (setq switch-window-shortcut-style 'qwerty)
  (setq switch-window-qwerty-shortcuts
        '("a" "s" "d" "f" "j" "k" "l" "i" "o"))
  :bind
  ([remap other-window] . switch-window))
\end{minted}

\subsection{Following window splits}
\label{sec:org5e0ee6e}

After you split a window, your focus remains in the previous one. This annoyed me so much I wrote these two, they take care of it.
\begin{minted}[frame=lines,linenos=true]{common-lisp}
(defun split-and-follow-horizontally ()
  (interactive)
  (split-window-below)
  (balance-windows)
  (other-window 1))
(global-set-key (kbd "C-x 2") 'split-and-follow-horizontally)

(defun split-and-follow-vertically ()
  (interactive)
  (split-window-right)
  (balance-windows)
  (other-window 1))
(global-set-key (kbd "C-x 3") 'split-and-follow-vertically)

\end{minted}



\subsection{Helm}
\label{sec:org08a78aa}

\begin{verbatim}
(use-package helm
  :ensure t
  :bind
  ("C-x C-f" . 'helm-find-files)
  ("C-x C-b" . 'helm-buffers-list)
  ("M-x" . 'helm-M-x)
  :config
  (defun daedreth/helm-hide-minibuffer ()
    (when (with-helm-buffer helm-echo-input-in-header-line)
      (let ((ov (make-overlay (point-min) (point-max) nil nil t)))
        (overlay-put ov 'window (selected-window))
        (overlay-put ov 'face
                     (let ((bg-color (face-background 'default nil)))
                       `(:background ,bg-color :foreground ,bg-color)))
        (setq-local cursor-type nil))))
  (add-hook 'helm-minibuffer-set-up-hook 'daedreth/helm-hide-minibuffer)
  (setq helm-autoresize-max-height 0
        helm-autoresize-min-height 40
        helm-M-x-fuzzy-match t
        helm-buffers-fuzzy-matching t
        helm-recentf-fuzzy-match t
        helm-semantic-fuzzy-match t
        helm-imenu-fuzzy-match t
        helm-split-window-in-side-p nil
        helm-move-to-line-cycle-in-source nil
        helm-ff-search-library-in-sexp t
        helm-scroll-amount 8 
        helm-echo-input-in-header-line t)
  :init
  (helm-mode 1))

(require 'helm-config)    
(helm-autoresize-mode 1)
(define-key helm-find-files-map (kbd "C-b") 'helm-find-files-up-one-level)
(define-key helm-find-files-map (kbd "C-f")
'helm-execute-persistent-action)

(global-set-key (kbd "M-y") 'helm-show-kill-ring)
(global-set-key (kbd "C-x b") 'helm-mini)

\end{verbatim}


\subsection{Avy}
\label{sec:org9a77ebf}

\begin{minted}[frame=lines,linenos=true]{common-lisp}
(use-package avy
  :ensure t
  :bind
  ("M-s" . avy-goto-char-timer))
\end{minted}

\subsection{Swoop}
\label{sec:orgdac3f42}
\begin{minted}[frame=lines,linenos=true]{common-lisp}

(use-package helm-swoop
  :bind (("C-c h o" . helm-swoop)
         ("C-c s" . helm-multi-swoop-all))
  :config
  ;; When doing isearch, hand the word over to helm-swoop
  (define-key isearch-mode-map (kbd "M-i") 'helm-swoop-from-isearch)

  ;; From helm-swoop to helm-multi-swoop-all
  (define-key helm-swoop-map (kbd "M-i") 'helm-multi-swoop-all-from-helm-swoop)

  ;; Save buffer when helm-multi-swoop-edit complete
  (setq helm-multi-swoop-edit-save t)

  ;; If this value is t, split window inside the current window
  (setq helm-swoop-split-with-multiple-windows t)

  ;; Split direcion. 'split-window-vertically or 'split-window-horizontally
  (setq helm-swoop-split-direction 'split-window-vertically)

  ;; If nil, you can slightly boost invoke speed in exchange for text color
  (setq helm-swoop-speed-or-color t))

\end{minted}

\subsection{Winner mode}
\label{sec:org057ae62}

\begin{minted}[frame=lines,linenos=true]{common-lisp}
(use-package winner
  :ensure t
  :init (winner-mode 1))
\end{minted}

\section{Completion}
\label{sec:org6d8fe5c}

\subsection{Recentf}
\label{sec:orgb9a1a8c}
\begin{minted}[frame=lines,linenos=true]{common-lisp}

     (use-package recentf
       :init
       (setq recentf-max-menu-items 25
             recentf-auto-cleanup 'never
             recentf-max-saved-items 50
             recentf-keep '(file-remote-p file-readable-p))
       (recentf-mode 1)
       (let ((last-ido "~/.emacs.d/ido.last"))
         (when (file-exists-p last-ido)
           (delete-file last-ido)))

       :bind ("C-c f r" . recentf-open-files))


(defun ido-recentf-open ()
  "Use `ido-completing-read' to \\[find-file] a recent file"
  (interactive)
  (if (find-file (ido-completing-read "Find recent file: " recentf-list))
      (message "Opening file...")
    (message "Aborting")))

(global-set-key (kbd "C-x C-r") 'ido-recentf-open)




\end{minted}
\subsection{IDO}
\label{sec:orge523921}

\begin{minted}[frame=lines,linenos=true]{common-lisp}
(use-package ido
  :ensure t
  :init  (setq ido-enable-flex-matching t
               ido-ignore-extensions t
               ido-use-virtual-buffers t
               ido-everywhere t)
  :config
  (ido-mode 1)
  (ido-everywhere 1)
  (add-to-list 'completion-ignored-extensions ".pyc"))

\end{minted}
ido-completing-read+
\begin{minted}[frame=lines,linenos=true]{common-lisp}
(use-package ido-completing-read+
  :ensure t
  :config
  (ido-ubiquitous-mode))
\end{minted}

FLX package
\begin{minted}[frame=lines,linenos=true]{common-lisp}
(use-package flx-ido
  :ensure t
  :init (setq ido-enable-flex-matching t
              ido-use-faces nil)
  :config (flx-ido-mode 1))

\end{minted}

Vertical mode 
\begin{minted}[frame=lines,linenos=true]{common-lisp}
(use-package ido-vertical-mode
  :ensure t
  :init               ; I like up and down arrow keys:
  (setq ido-vertical-define-keys 'C-n-C-p-up-and-down)
  :config
  (ido-vertical-mode 1))

\end{minted}

\begin{minted}[frame=lines,linenos=true]{common-lisp}
(defun ido-sort-mtime ()
  "Reorder the IDO file list to sort from most recently modified."
  (setq ido-temp-list
        (sort ido-temp-list
              (lambda (a b)
                (ignore-errors
                  (time-less-p
                   (sixth (file-attributes (concat ido-current-directory b)))
                   (sixth (file-attributes (concat ido-current-directory a))))))))
  (ido-to-end  ;; move . files to end (again)
   (delq nil (mapcar
              (lambda (x) (and (char-equal (string-to-char x) ?.) x))
              ido-temp-list))))

(add-hook 'ido-make-file-list-hook 'ido-sort-mtime)
(add-hook 'ido-make-dir-list-hook 'ido-sort-mtime)

\end{minted}

\subsection{SMEX}
\label{sec:org684248b}
\begin{minted}[frame=lines,linenos=true]{common-lisp}
(use-package smex
  :ensure t
  :init (smex-initialize)
  :bind ("M-x" . smex)
  ("M-X" . smex-major-mode-commands))

\end{minted}
\subsection{Ivy}
\label{sec:org8242f75}
\begin{minted}[frame=lines,linenos=true]{common-lisp}

(use-package ivy
  :ensure t
  :config
  (setq ivy-use-virtual-buffers t
        enable-recursive-minibuffers t
        ivy-wrap t 
        ivy-count-format "%d/%d "))
\end{minted}


\subsection{Counsel}
\label{sec:orgd29dc36}
Counsel tramp 
\begin{minted}[frame=lines,linenos=true]{common-lisp}
(use-package counsel-tramp
  :ensure t
  )
\end{minted}

\section{Writing}
\label{sec:orgafea673}

\subsection{Flyspell config}
\label{sec:orgd3fdf17}

Installing aspell on linux:

\begin{verbatim}
apt install aspell aspell-en
\end{verbatim}

on mac: 

\begin{verbatim}
brew install aspell 
\end{verbatim}

Note in the config below I assume aspell is installed in \texttt{/usr/bin/} !. 

\begin{minted}[frame=lines,linenos=true]{common-lisp}
(use-package flyspell
  :ensure t
  :diminish flyspell-mode
  :init
  (add-hook 'prog-mode-hook 'flyspell-prog-mode)

  (dolist (hook '(text-mode-hook org-mode-hook))
    (add-hook hook (lambda () (flyspell-mode 1))))

  (dolist (hook '(change-log-mode-hook log-edit-mode-hook org-agenda-mode-hook))
    (add-hook hook (lambda () (flyspell-mode -1))))

  :config
  (setq ispell-program-name "aspell"
        ispell-local-dictionary "en_GB"
        ;;ispell-dictionary "american" ; better for aspellr
        ispell-extra-args '("--sug-mode=ultra" "--lang=en_GB")
        ispell-list-command "--list"
        ispell-local-dictionary-alist '(("en_GB" "[[:alpha:]]" "[^[:alpha:]]" "['‘’]"
                                         t ; Many other characters
                                         ("-d" "en_GB") nil utf-8))))

\end{minted}

There is more stuff in Howard Abram's config but I'll leave this for now..



\subsection{Writegood mode}
\label{sec:org2be6598}
This does not work - there is a wring gpg signature in melpa\ldots{} 

\begin{minted}[frame=lines,linenos=true]{common-lisp}

(when (file-exists-p "/home/user/programs/writegood-mode")
  (message "Loading writegood-mode")
  (add-to-list 'load-path "/home/user/programs/writegood-mode")
  (require 'writegood-mode)
  (add-hook 'text-mode-hook 'writegood-mode)
  (add-hook 'org-mode-hook 'writegood-mode)
  )
\end{minted}

this mode will improve various aspects of writing. 

end.


\subsection{LangTool}
\label{sec:org1401921}

I added the Emacs-langtool code from:

\url{https://github.com/mhayashi1120/Emacs-langtool}

To my \texttt{/elisp/} directory. 

To install langtool install \texttt{maven} package, java 8 then:

\begin{verbatim}
cd ~/programs
git clone https://github.com/languagetool-org/languagetool.git
./build.sh languagetool-standalone package

\end{verbatim}
This does not work! 

I now simply download the pre-compiles zip package\ldots{} 

To load: 
\begin{minted}[frame=lines,linenos=true]{common-lisp}
(require 'langtool)
(setq langtool-language-tool-jar "/home/user/programs/langtool/LanguageTool-4.0/languagetool-commandline.jar")
\end{minted}

\section{Org-mode}
\label{sec:org47caeac}
\subsection{General setup}
\label{sec:orga41f40f}

load org mode

\begin{minted}[frame=lines,linenos=true]{common-lisp}
(use-package org
  :init
  (setq org-use-speed-commands t
        org-return-follows-link t
        org-completion-use-ido t
        org-outline-path-complete-in-steps nil))
\end{minted}

Directory, inbox ..

\begin{minted}[frame=lines,linenos=true]{common-lisp}
(setq org-directory "~/")
(defun org-file-path (filename)
  "Return the absolute address of an org file, given its relative name."
  (concat (file-name-as-directory org-directory) filename))
(setq org-index-file (org-file-path "/capture/inbox.org"))
(setq org-archive-location
      (concat (org-file-path "archive.org") "::* From %s"))
\end{minted}

Multiple files for agenda source:

\begin{minted}[frame=lines,linenos=true]{common-lisp}
;;   (setq org-agenda-files (list org-index-file))
(setq org-agenda-files '("~/capture"
                         "~/work"
                         "~/planning"
                         "~/life"))
\end{minted}

Refile targets / create new targets if necessary

\begin{minted}[frame=lines,linenos=true]{common-lisp}
;;(setq org-refile-targets '((org-agenda-files :maxlevel . 3)))
(setq org-refile-targets '(("~/work/work-todo.org" :maxlevel . 2)
                           ("~/work/work-todo-archive.org" :maxlevel . 2)
                           ("~/life/life-todo.org" :maxlevel . 2)
                           ))
(setq org-refile-use-outline-path 'file)
(setq org-refile-allow-creating-parent-nodes 'confirm)
(setq org-refile-allow-creating-parent-nodes 'confirm)
\end{minted}

Drawers 
\begin{minted}[frame=lines,linenos=true]{common-lisp}
(setq org-log-into-drawer t)

;; Add the REPORT drawer
(setq org-drawers '("PROPERTIES" "CLOCK" "LOGBOOK" "REPORT"))
\end{minted}

Hitting C-c C-x C-s will mark a todo as done and move it to an appropriate place
in the archive.

\begin{minted}[frame=lines,linenos=true]{common-lisp}
(defun tl/mark-done-and-archive ()
  "Mark the state of an org-mode item as DONE and archive it."
  (interactive)
  (org-todo 'done)
  (org-archive-subtree))

;;    (define-key org-mode-map (kbd "C-c C-x C-s") 'tl/mark-done-and-archive)



\end{minted}

Record the time that a todo was archived.

\begin{minted}[frame=lines,linenos=true]{common-lisp}
(setq org-log-done 'time)
\end{minted}

\begin{minted}[frame=lines,linenos=true]{common-lisp}
(add-hook 'org-mode-hook 'visual-line-mode)
\end{minted}
\subsection{Capture}
\label{sec:orgbb5c883}
Capture templates..
\begin{minted}[frame=lines,linenos=true]{common-lisp}
(setq org-capture-templates
      (quote (("t" "todo" entry (file+headline org-index-file "Inbox")
               "* TODO %?\nSCHEDULED: %(org-insert-time-stamp (org-read-date nil t \"+0d\"))\n%a\n")
              ("n" "note" entry (file+headline org-index-file "Inbox")
               "* %?\n\n  %i\n\n  See: %a" :empty-lines 1)
              ("r" "respond" entry (file+headline org-index-file "Inbox")
               "* TODO Respond to %:from on %:subject\nSCHEDULED: %(org-insert-time-stamp (org-read-date nil t \"+0d\"))\n%a\n")
              ("m" "Mail" entry (file+headline org-index-file "Inbox")
               "* TODO %?\n%a   %:from %:fromname %:fromaddress" :prepend t :jump-to-captured t)
              ("p" "Daily Plan" plain (file+datetree "~/planning/daily-plan.org")
               "+ [ ] The 3 most important tasks [/]
                - [ ] 
                - [ ] 
                - [ ] 
              + [ ] Other tasks that are in the system [/]
                - [ ] 
              + [ ] ToDos which are not tracked by my system [/]
                - [ ] " :immediate-finish t)
              )))
\end{minted}

\subsection{Taking Meeting Notes}
\label{sec:org6ec902f}

directly from \url{https://github.com/howardabrams/dot-files/blob/master/emacs-org.org})

I’ve notice that while I really like taking notes in a meeting, I don’t always like the multiple windows I have opened, so I created this function that I can easily call to eliminate distractions during a meeting.
\begin{minted}[frame=lines,linenos=true]{common-lisp}

(defun meeting-notes ()
  "Call this after creating an org-mode heading for where the notes for the meeting
should be. After calling this function, call 'meeting-done' to reset the environment."
  (interactive)
  (outline-mark-subtree)                              ;; Select org-mode section
  (narrow-to-region (region-beginning) (region-end))  ;; Only show that region
  (deactivate-mark)
  (delete-other-windows)                              ;; Get rid of other windows
  (text-scale-set 3)                                  ;; Text is now readable by others
  (fringe-mode 0)
  (message "When finished taking your notes, run meeting-done."))

\end{minted}
Of course, I need an ‘undo’ feature when the meeting is over…
\begin{minted}[frame=lines,linenos=true]{common-lisp}
(defun meeting-done ()
  "Attempt to 'undo' the effects of taking meeting notes."
  (interactive)
  (widen)                                       ;; Opposite of narrow-to-region
  (text-scale-set 0)                            ;; Reset the font size increase
  (fringe-mode 1)
  (winner-undo))                                ;; Put the windows back in place

\end{minted}

End.

\subsection{Coding}
\label{sec:orge589d57}

Allow babel to evaluate C \ldots{}

\begin{minted}[frame=lines,linenos=true]{common-lisp}
(org-babel-do-load-languages
 'org-babel-load-languages
 '((C . t)
   (R . t)
   (dot . t)
   (emacs-lisp . t)
   (shell . t) 
   (awk . t)
   (makefile . t)
   (latex . t)
   (java . t)
   (clojure . t)
   ))

\end{minted}

Don’t ask before evaluating code blocks.
\begin{minted}[frame=lines,linenos=true]{common-lisp}

(setq org-confirm-babel-evaluate nil)

\end{minted}

smart brackets in export

\begin{minted}[frame=lines,linenos=true]{common-lisp}
(setq org-export-with-smart-quotes t)
\end{minted}

Done.
\subsection{Export}
\label{sec:orgc9485bb}

Export packages\ldots{}

\begin{minted}[frame=lines,linenos=true]{common-lisp}
(require 'ox-latex)
(require 'ox-beamer)
\end{minted}

Htmlize required for reveal\ldots{}

\begin{minted}[frame=lines,linenos=true]{common-lisp}
(use-package htmlize
  :ensure t)
\end{minted}

Use minted package for code: 

\begin{minted}[frame=lines,linenos=true]{common-lisp}
(setq org-latex-listings 'minted)
(setq org-latex-minted-options
      '(("frame" "lines") ("linenos=true")))

\end{minted}

\subsection{Flyspell}
\label{sec:org90a861a}
Enable spell-checking in Org-mode.
\begin{minted}[frame=lines,linenos=true]{common-lisp}
(add-hook 'org-mode-hook 'flyspell-mode)
\end{minted}

\subsection{Color and display}
\label{sec:org1f3b81d}

Use syntax highlighting in source blocks while editing.
\begin{minted}[frame=lines,linenos=true]{common-lisp}
(setq org-src-fontify-natively t)
\end{minted}

Make TAB act as if it were issued in a buffer of the language’s major mode.
\begin{minted}[frame=lines,linenos=true]{common-lisp}
(setq org-src-tab-acts-natively t)
\end{minted}

When editing a code snippet, use the current window rather than popping open a
new one (which shows the same information).
\begin{minted}[frame=lines,linenos=true]{common-lisp}
(setq org-src-window-setup 'current-window)
\end{minted}

\subsection{Image preview}
\label{sec:orgff170d2}

Inline images support:

\begin{minted}[frame=lines,linenos=true]{common-lisp}
(setq org-latex-create-formula-image-program 'imagemagick)

(add-to-list 'org-latex-packages-alist
             '("" "tikz" t))

(eval-after-load "preview"
  '(add-to-list 'preview-default-preamble "\\PreviewEnvironment{tikzpicture}" t))
(setq org-latex-create-formula-image-program 'imagemagick)


(setq org-confirm-babel-evaluate nil)
(add-hook 'org-babel-after-execute-hook 'org-display-inline-images)   
(add-hook 'org-mode-hook 'org-display-inline-images)
\end{minted}

\subsection{Keybindings}
\label{sec:orga45f1d2}


Standard bindings

\begin{minted}[frame=lines,linenos=true]{common-lisp}
(define-key global-map "\C-cl" 'org-store-link)
(define-key global-map "\C-ca" 'org-agenda)
(define-key global-map "\C-cc" 'org-capture)
\end{minted}

Quickly open index file
\begin{minted}[frame=lines,linenos=true]{common-lisp}
(defun open-index-file ()
  "Open the master org TODO list."
  (interactive)
  (find-file org-index-file)
  (flycheck-mode -1)
  (end-of-buffer))

(global-set-key (kbd "C-c i") 'open-index-file)
\end{minted}


undef a key

\begin{minted}[frame=lines,linenos=true]{common-lisp}
(add-hook 'org-mode-hook
          '(lambda ()
             ;; Undefine C-c [ and C-c ] since this breaks my
             ;; org-agenda files when directories are include It
             ;; expands the files in the directories individually
             (org-defkey org-mode-map "\C-c[" 'undefined))
          'append)

\end{minted}

\subsection{Org-ref}
\label{sec:orgaa338a4}

\begin{minted}[frame=lines,linenos=true]{common-lisp}
(use-package org-ref
  :ensure t)
\end{minted}

Make \texttt{supercite} the default citation type:
\begin{minted}[frame=lines,linenos=true]{common-lisp}
(setq org-ref-default-citation-link "supercite")
\end{minted}


Define format for bibtex entries


\begin{minted}[frame=lines,linenos=true]{common-lisp}

;; variables that control bibtex key format for auto-generation
;; I want firstauthor-year-title-words
;; this usually makes a legitimate filename to store pdfs under.
(setq bibtex-autokey-year-length 4
      bibtex-autokey-name-year-separator "-"
      bibtex-autokey-year-title-separator "-"
      bibtex-autokey-titleword-separator "-"
      bibtex-autokey-titlewords 2
      bibtex-autokey-titlewords-stretch 1
      bibtex-autokey-titleword-length 5)
\end{minted}

Where are the refs?

\begin{minted}[frame=lines,linenos=true]{common-lisp}
(setq reftex-default-bibliography '("~/work/bibliography/references.bib"))

;; see org-ref for use of these variables
(setq org-ref-bibliography-notes "~/work/bibliography/notes.org"
      org-ref-default-bibliography '("~/work/bibliography/references.bib")
      org-ref-pdf-directory "~/work/bibliography/bibtex-pdfs/")



\end{minted}

\begin{minted}[frame=lines,linenos=true]{common-lisp}
(setq org-ref-completion-library 'org-ref-ivy-cite)

\end{minted}
End.

\subsection{Latex templates}
\label{sec:org6f6bf4b}
Latex templates
\begin{minted}[frame=lines,linenos=true]{common-lisp}



;;(setq org-latex-to-pdf-process '("xelatex %f && bibtex %f && xelatex %f && xelatex %f"))
(defun sk-latexmk-cmd (backend)
  "When exporting from .org with latex, automatically run latex,
     pdflatex, or xelatex as appropriate, using latexmk."
  (when (org-export-derived-backend-p backend 'latex)
    (let ((texcmd)))
    ;; default command: xelatex
    (setq texcmd "jobname=$(basename %f | sed 's/\.tex//');latexmk -xelatex -shell-escape -quiet %f && mkdir -p latex.d && mv ${jobname}.* latex.d/. && mv latex.d/${jobname}.{org,pdf,fdb_latexmk,aux} .")
    ;; pdflatex -> .pdf
    (if (string-match "LATEX_CMD: pdflatex" (buffer-string))
        (setq texcmd "latexmk -pdflatex='pdflatex -shell-escape -interaction nonstopmode' -pdf -bibtex -f %f"))
    ;; xelatex -> .pdf
    (if (string-match "LATEX_CMD: xelatex" (buffer-string))
        (setq texcmd "latexmk -pdflatex='xelatex -shell-escape -interaction nonstopmode' -pdf -bibtex -f  %f"))
    ;; LaTeX compilation command
    (setq org-latex-pdf-process (list texcmd))))

(org-add-hook 'org-export-before-processing-hook 'sk-latexmk-cmd)

(unless (boundp 'org-latex-classes)
  (setq org-latex-classes nil))
\end{minted}

\subsection{CV}
\label{sec:org403552c}

\begin{minted}[frame=lines,linenos=true]{common-lisp}
(add-to-list 'org-latex-classes
             '("CV"
               "\\documentclass[11pt]{article}
     \\usepackage{\\string~\"/.emacs.d/latex_templates/cv\"}
     [NO-DEFAULT-PACKAGES]
     [NO-PACKAGES]"
               ("\\section{%s}" . "\\section*{%s}")
               ("\\subsection{%s}" . "\\subsection*{%s}")
               ("\\subsubsection{%s}" . "\\subsubsection*{%s}")
               ("\\paragraph{%s}" . "\\paragraph*{%s}")
               ("\\subparagraph{%s}" . "\\subparagraph*{%s}")))
\end{minted}

\subsection{NHMRC project grant}
\label{sec:orgb1cd690}

\begin{minted}[frame=lines,linenos=true]{common-lisp}
(add-to-list 'org-latex-classes
             '("NHMRC_project_grant"
               "\\documentclass[12pt,table,names]{article}
\\usepackage{\\string~\"/.emacs.d/latex_templates/NHMRC_grant\"}
[NO-DEFAULT-PACKAGES]
[NO-PACKAGES]"
               ("\\section{%s}" . "\\section*{%s}")
               ("\\subsection{%s}" . "\\subsection*{%s}")
               ("\\subsubsection{%s}" . "\\subsubsection*{%s}")
               ("\\paragraph{%s}" . "\\paragraph*{%s}")
               ("\\subparagraph{%s}" . "\\subparagraph*{%s}")))
\end{minted}
Rebuttal\ldots{} 
\begin{minted}[frame=lines,linenos=true]{common-lisp}
(add-to-list 'org-latex-classes
             '("NHMRC_project_grant_rebuttal"
               "\\documentclass[12pt,table,names]{article}
  \\usepackage{\\string~\"/.emacs.d/latex_templates/NHMRC_grant\"}
  [NO-DEFAULT-PACKAGES]
  [NO-PACKAGES]"
               ("\\subsection{%s}" . "\\section*{%s}")
               ("\\subsubsection{%s}" . "\\subsection*{%s}")q
               ("\\subsubsection{%s}" . "\\subsubsection*{%s}")
               ("\\paragraph{%s}" . "\\paragraph*{%s}")
               ("\\subparagraph{%s}" . "\\subparagraph*{%s}")))

\end{minted}

\subsection{NHMRC Investigator}
\label{sec:orgef76cb5}

\begin{minted}[frame=lines,linenos=true]{common-lisp}
(add-to-list 'org-latex-classes
             '("NHMRC_investigator_grant"
               "\\documentclass[12pt,table,names]{article}
\\usepackage{\\string~\"/.emacs.d/latex_templates/NHMRC_investigator\"}
[NO-DEFAULT-PACKAGES]
[NO-PACKAGES]"
               ("\\section{%s}" . "\\section*{%s}")
               ("\\subsection{%s}" . "\\subsection*{%s}")
               ("\\subsubsection{%s}" . "\\subsubsection*{%s}")
               ("\\paragraph{%s}" . "\\paragraph*{%s}")
               ("\\subparagraph{%s}" . "\\subparagraph*{%s}")))
\end{minted}

\subsection{ARC Discovery Grant}
\label{sec:org47304dc}

Main grant 
\begin{minted}[frame=lines,linenos=true]{common-lisp}
(add-to-list 'org-latex-classes
             '("ARC_discovery_grant"
               "\\documentclass[12pt]{article}
\\usepackage{\\string~\"/.emacs.d/latex_templates/ARC_discovery\"}
[NO-DEFAULT-PACKAGES]
[NO-PACKAGES]"
               ("\\section{%s}" . "\\section*{%s}")
               ("\\subsection{%s}" . "\\subsection*{%s}")
               ("\\subsubsection{%s}" . "\\subsubsection*{%s}")
               ("\\paragraph{%s}" . "\\paragraph*{%s}")))
\end{minted}

Special formatting for the ROPE sections.

\begin{minted}[frame=lines,linenos=true]{common-lisp}
(add-to-list 'org-latex-classes
             '("ARC_ROPE"
               "\\documentclass[12pt]{article}
\\usepackage{\\string~\"/.emacs.d/latex_templates/ARC_discovery_ROPE\"}
[NO-DEFAULT-PACKAGES]
[NO-PACKAGES]"
               ("\\section{%s}" . "\\section*{%s}")
               ("\\subsection{%s}" . "\\subsection*{%s}")
               ("\\subsubsection{%s}" . "\\subsubsection*{%s}")
               ("\\paragraph{%s}" . "\\paragraph*{%s}")))
\end{minted}



\subsection{Nature style paper}
\label{sec:org866e518}

\begin{minted}[frame=lines,linenos=true]{common-lisp}
(add-to-list 'org-latex-classes '("naturedef"
                                  "\\documentclass[fleqn,10pt]{wlscirep}
 [NO-DEFAULT-PACKAGES]
 [PACKAGES]
 [EXTRA]"
                                  ("\\section{%s}" . "\\section*{%s}")
                                  ("\\subsection{%s}" . "\\subsection*{%s}")
                                  ("\\subsubsection{%s}" . "\\subsubsection*{%s}")
                                  ("\\paragraph{%s}" . "\\paragraph*{%s}")
                                  ("\\subparagraph{%s}" . "\\subparagraph*{%s}")))
\end{minted}

\begin{minted}[frame=lines,linenos=true]{common-lisp}
(add-to-list 'org-latex-classes
             '("nature"
               "\\documentclass[12pt]{article}
     \\usepackage{\\string~\"/.emacs.d/latex_templates/nature\"}
     [NO-DEFAULT-PACKAGES]
     [NO-PACKAGES]"
               ("\\section*{%s}" . "\\section*{%s}")
               ("\\subsection{%s}" . "\\subsection*{%s}")
               ("\\subsubsection{%s}" . "\\subsubsection*{%s}")
               ("\\paragraph{%s}" . "\\paragraph*{%s}")
               ("\\subparagraph{%s}" . "\\subparagraph*{%s}")))
\end{minted}

\subsection{Bioinformatics paper}
\label{sec:orgf8058ed}

\begin{minted}[frame=lines,linenos=true]{common-lisp}
(add-to-list 'org-latex-classes '("bioinfo"
                                  "\\documentclass{bioinfo}
 [NO-DEFAULT-PACKAGES]
 [PACKAGES]
 [EXTRA]"
                                  ("\\section{%s}" . "\\section*{%s}")
                                  ("\\subsection{%s}" . "\\subsection*{%s}")
                                  ("\\subsubsection{%s}" . "\\subsubsection*{%s}")
                                  ("\\paragraph{%s}" . "\\paragraph*{%s}")
                                  ("\\subparagraph{%s}" . "\\subparagraph*{%s}")))
\end{minted}


\subsection{Internal report}
\label{sec:orga316e65}
\begin{minted}[frame=lines,linenos=true]{common-lisp}
(add-to-list 'org-latex-classes
             '("report"
               "\\documentclass[12pt]{article}
\\usepackage{\\string~\"/.emacs.d/latex_templates/report\"}
[NO-DEFAULT-PACKAGES]
[NO-PACKAGES]"
               ("\\section{%s}" . "\\section*{%s}")
               ("\\subsection{%s}" . "\\subsection*{%s}")
               ("\\subsubsection{%s}" . "\\subsubsection*{%s}")
               ("\\paragraph{%s}" . "\\paragraph*{%s}")
               ("\\subparagraph{%s}" . "\\subparagraph*{%s}")))
\end{minted}

\subsection{Simple presentation}
\label{sec:org1a12949}

\begin{minted}[frame=lines,linenos=true]{common-lisp}
(add-to-list 'org-latex-classes
             '("simplepresentation"
               "\\documentclass[aspectratio=169,18pt,t]{beamer}
\\usepackage{\\string~\"/.emacs.d/latex_templates/simple\"}
[NO-DEFAULT-PACKAGES]
[NO-PACKAGES]"
               ("\\section{%s}" . "\\section*{%s}")
               ("\\begin{frame}[fragile]\\frametitle{%s}"
                "\\end{frame}"
                "\\begin{frame}[fragile]\\frametitle{%s}"
                "\\end{frame}")))
\end{minted}

\subsection{Fancier presentation}
\label{sec:org13dced1}

\begin{minted}[frame=lines,linenos=true]{common-lisp}

(add-to-list 'org-latex-classes
             '("modernpresentation"
               "\\documentclass[14pt]{beamer}
    \\usepackage{\\string~\"/.emacs.d/latex_templates/modern\"}
    [NO-DEFAULT-PACKAGES]
    [NO-PACKAGES]"
               ("\\section{%s}" . "\\section*{%s}")
               ("\\begin{frame}[fragile]\\frametitle{%s}"
                "\\end{frame}")))

\end{minted}
end. 

\section{Programming}
\label{sec:orgb80ae5b}

General programming settings..

\subsection{General}
\label{sec:org34238b5}

I like shallow indentation, but tabs are displayed as 8 characters by default. This reduces that.

\begin{minted}[frame=lines,linenos=true]{common-lisp}
(setq-default tab-width 2)
\end{minted}

Treating terms in CamelCase symbols as separate words makes editing a little
easier for me, so I like to use subword-mode everywhere.
\begin{minted}[frame=lines,linenos=true]{common-lisp}
(global-subword-mode 1)
\end{minted}

Compilation output goes to the \textbf{compilation} buffer. I rarely have that window
selected, so the compilation output disappears past the bottom of the window.
This automatically scrolls the compilation window so I can always see the
output.

\begin{minted}[frame=lines,linenos=true]{common-lisp}

;;(setq compilation-scroll-output t)
(setq compile-command "make")
(setq compilation-scroll-output 'first-error)
(setq compilation-always-kill t)
(setq compilation-disable-input t)
(add-hook 'compilation-mode-hook 'visual-line-mode)

\end{minted}

Flycheck 
\begin{minted}[frame=lines,linenos=true]{common-lisp}
(use-package flycheck
  :ensure t
  :init
  (add-hook 'after-init-hook 'global-flycheck-mode)
  (add-hook 'c-mode-hook (lambda () (setq flycheck-clang-language-standard "c11")))
  :config
  (setq-default flycheck-disabled-checkers '(emacs-lisp-checkdoc)))

\end{minted}

\subsection{Line Numbering}
\label{sec:orgb2d9e82}

\begin{minted}[frame=lines,linenos=true]{common-lisp}
(use-package linum-relative
  :ensure t
  :config
  (setq linum-relative-current-symbol "")
  (add-hook 'prog-mode-hook 'linum-relative-mode))
\end{minted}

\subsection{Magit}
\label{sec:orgcfaee21}

I played with this before.. 

\begin{minted}[frame=lines,linenos=true]{common-lisp}
(use-package magit
  :ensure t
  :commands magit-status magit-blame
  :init
  (defadvice magit-status (around magit-fullscreen activate)
    (window-configuration-to-register :magit-fullscreen)
    ad-do-it
    (delete-other-windows))
  :config
  (setq magit-branch-arguments nil
        ;; use ido to look for branches
        magit-completing-read-function 'magit-ido-completing-read
        ;; don't put "origin-" in front of new branch names by default
        magit-default-tracking-name-function 'magit-default-tracking-name-branch-only
        magit-push-always-verify nil
        ;; Get rid of the previous advice to go into fullscreen
        magit-restnore-window-configuration t)

  :bind ("C-x g" . magit-status))

\end{minted}

magit end. 
\subsection{Aggressive Auto Indention}
\label{sec:orgb88a9cc}

Automatically indent without use of the tab found in this article, and seems to be quite helpful for many types of programming languages.

To begin, we create a function that can indent a function by calling indent-region on the beginning and ending points of a function.
\begin{minted}[frame=lines,linenos=true]{common-lisp}
(defun indent-defun ()
  "Indent current defun.
Do nothing if mark is active (to avoid deactivaing it), or if
buffer is not modified (to avoid creating accidental
modifications)."
  (interactive)
  (unless (or (region-active-p)
              buffer-read-only
              (null (buffer-modified-p)))
    (let ((l (save-excursion (beginning-of-defun 1) (point)))
          (r (save-excursion (end-of-defun 1) (point))))
      (cl-letf (((symbol-function 'message) #'ignore))
        (indent-region l r)))))
\end{minted}

Next, create a hook that will call the indent-defun with every command call:

\begin{minted}[frame=lines,linenos=true]{common-lisp}
(defun activate-aggressive-indent ()
  "Locally add `ha/indent-defun' to `post-command-hook'."
  (add-hook 'post-command-hook
            'indent-defun nil 'local))
\end{minted}

\subsection{Auto Complete}
\label{sec:org2882157}
\begin{minted}[frame=lines,linenos=true]{common-lisp}
(use-package company-c-headers
  :ensure t
  )

(use-package company-math 
  :ensure t
  )

(use-package company-shell 
  :ensure t
  )

(use-package company
  :ensure t
  :init
  (setq company-dabbrev-ignore-case t
        company-show-numbers t)
  (add-hook 'after-init-hook 'global-company-mode)
  :config
  (setq company-idle-delay 0.05)
  (setq company-minimum-prefix-length 3)
  (setq company-tooltip-align-annotations t)

  (add-to-list 'company-backends 'company-math-symbols-unicode)
  (add-to-list 'company-backends 'company-c-headers)
  ;;(add-to-list 'company-backends 'company-shell)
  :bind ("C-:" . company-complete)  ; In case I don't want to wait
  :diminish company-mode)

(use-package company-quickhelp
  :ensure t
  :config
  (company-quickhelp-mode 1))

(add-hook 'c-mode-hook 'company-mode)

(use-package company-statistics
  :ensure t 
  )

(with-eval-after-load 'company
  (define-key company-active-map (kbd "M-n") nil)
  (define-key company-active-map (kbd "M-p") nil)
  (define-key company-active-map (kbd "C-n") #'company-select-next)
  (define-key company-active-map (kbd "C-p") #'company-select-previous)
  (define-key company-active-map (kbd "SPC") #'company-abort))

\end{minted}

To make this work properly, I need to manually specify the include paths by
putting a \texttt{.dir-locals.el} into the source directory of my C code. I.e. most
of the time this will be \texttt{src} and I need to point to
\texttt{../tldevel}. 

In addition add the include path to flycheck-clang! 

\begin{verbatim}
((c-mode (eval setq company-clang-arguments (append company-clang-arguments '("-I../tldevel")))))
((c-mode (eval setq  flycheck-clang-include-path (append  flycheck-clang-include-path '("-I../tldevel")))))      
\end{verbatim}

\subsection{hippie expand}
\label{sec:orgca17fef}


\begin{minted}[frame=lines,linenos=true]{common-lisp}
(global-set-key (kbd "M-/") 'hippie-expand)

(setq hippie-expand-try-functions-list
      '(try-expand-dabbrev
        try-expand-dabbrev-all-buffers
        try-expand-dabbrev-from-kill
        try-complete-file-name-partially
        try-complete-file-name
        try-expand-all-abbrevs
        try-expand-list
        try-expand-line))
\end{minted}

\subsection{Yasnippet}
\label{sec:org3bca3d8}

\begin{minted}[frame=lines,linenos=true]{common-lisp}
(use-package yasnippet
  :ensure t
  :diminish
  :init
  (yas-global-mode 1)
  :config
  (use-package yasnippet-snippets
    :ensure t)
  (yas-reload-all));
\end{minted}

\subsection{Comments}
\label{sec:orgd101916}
\begin{minted}[frame=lines,linenos=true]{common-lisp}
(use-package smart-comment
  :ensure t
  :bind ("M-;" . smart-comment))
\end{minted}

\subsection{C}
\label{sec:org59716f6}
\subsubsection{compile}
\label{sec:org158cbe8}

\begin{minted}[frame=lines,linenos=true]{common-lisp}
(global-set-key (kbd "<f5>") (lambda ()
                               (interactive)
                               (setq-local compilation-read-command nil)
                               (call-interactively 'compile)))

\end{minted}

Follow compilation

\begin{minted}[frame=lines,linenos=true]{common-lisp}
(setq compilation-scroll-output t)
\end{minted}



\subsubsection{ggtags}
\label{sec:orgd4fa8a4}


\begin{minted}[frame=lines,linenos=true]{common-lisp}
(use-package ggtags
  :ensure t
  :init
  (add-hook 'c-mode-common-hook
            (lambda ()
              (when (derived-mode-p 'c-mode)
                (ggtags-mode 1))))
  :config

                                        ; This must be set to the location of gtags (global)
  ;;(setq ggtags-executable-directory "~/global-6.5.6/bin/")
                                        ; Allow very large database files
  (setq ggtags-oversize-limit 104857600)
  (setq ggtags-sort-by-nearness t)
  (setq ggtags-use-idutils t)
  (setq ggtags-use-project-gtagsconf nil)

  :bind (
         ;;("M-," . gtags-pop-stack)
         ;; ("M-/" . ggtags-find-reference)
         ;;("M-]" . ggtags-idutils-query)

         :map ggtags-navigation-map
                                        ;Ergo
         ("M-u" . ggtags-navigation-previous-file)
         ("M-o" . ggtags-navigation-next-file)
         ("M-l" . ggtags-navigation-visible-mode)
         ("M-j" . ggtags-navigation-visible-mode)
         ("M-k" . next-error)
         ("M-i" . previous-error)
         ) ; end :bind
  )

\end{minted}

\subsubsection{Counsel-gtags}
\label{sec:orge97d4ad}

\begin{minted}[frame=lines,linenos=true]{common-lisp}
(use-package counsel-gtags
  :ensure t
  ;;:bind (
  ;;   ("M-t" . counsel-gtags-find-definition)
  ;; ("M-r" . counsel-gtags-find-reference)
  ;;("M-s" . counsel-gtags-find-symbol)
  ;; ("M-," . counsel-gtags-go-backward)
  ;; )
  :init 
  (add-hook 'c-mode-hook 'counsel-gtags-mode)
  (add-hook 'c++-mode-hook 'counsel-gtags-mode)
  )

\end{minted}

\subsubsection{Smartparens}
\label{sec:org90adffa}

\begin{minted}[frame=lines,linenos=true]{common-lisp}
(use-package smartparens
  :ensure t
  :config
  (sp-pair "'" nil :actions :rem)
  (sp-pair "`" nil :actions :rem)
  :init (add-hook 'c-mode-hook 'smartparens-mode))
\end{minted}

\subsubsection{smart scan}
\label{sec:orgaa3986c}

\begin{minted}[frame=lines,linenos=true]{common-lisp}
(use-package smartscan
  :ensure t
  :bind ("M-n" . smartscan-symbol-go-forward)
  ("M-p" . smartscan-symbol-go-backward))

\end{minted}

\subsection{Indenting}
\label{sec:org56dda6d}

SET BSD indent style

\begin{minted}[frame=lines,linenos=true]{common-lisp}

(add-hook 'c-mode-hook
          '(lambda()
             (c-set-style "bsd")
             ))
\end{minted}

Whitespace mode

\begin{minted}[frame=lines,linenos=true]{common-lisp}
(global-set-key (kbd "C-c w") 'whitespace-mode)
(add-hook 'prog-mode-hook (lambda () (interactive) (setq show-trailing-whitespace 1)))
\end{minted}

\begin{minted}[frame=lines,linenos=true]{common-lisp}
(setq-default indent-tabs-mode nil)
\end{minted}

\subsubsection{clean aindent mode}
\label{sec:org9590c9e}

\begin{minted}[frame=lines,linenos=true]{common-lisp}
(use-package clean-aindent-mode
  :ensure t 
  :init 
  (add-hook 'prog-mode-hook 'clean-aindent-mode)
  )
\end{minted}

\subsubsection{dtrt indent mode}
\label{sec:org080ab8e}

\begin{minted}[frame=lines,linenos=true]{common-lisp}
(use-package dtrt-indent 
  :ensure t
  :init 
  (dtrt-indent-mode 1)
  (setq dtrt-indent-verbosity 0)
  )
\end{minted}

\subsubsection{Whitespace bulter}
\label{sec:org49cbe33}

\begin{minted}[frame=lines,linenos=true]{common-lisp}
(use-package ws-butler
  :ensure t 
  :init
  (add-hook 'c-mode-common-hook 'ws-butler-mode)
  )
\end{minted}


\subsection{Don't ask for permission to run make}
\label{sec:orgd286755}
Don't ask with make command to run\ldots{}
\begin{minted}[frame=lines,linenos=true]{common-lisp}
(setq compilation-read-command nil)
\end{minted}

\subsection{ESS  (emacs speaks statistics\ldots{} ) and R}
\label{sec:orge796ab4}

I prefer to have my own R installation locally (see Rinstallation.org). Let's let emacs know about this: 

\begin{minted}[frame=lines,linenos=true]{common-lisp}
(setq-default inferior-R-program-name "/home/user/bin/R")
\end{minted}

Default ESS config: 
\begin{minted}[frame=lines,linenos=true]{common-lisp}
(use-package ess
  :ensure t
  :init (require 'ess-site))
\end{minted}


\section{Email}
\label{sec:org386beb2}


\begin{minted}[frame=lines,linenos=true]{common-lisp}
(require 'starttls)
(setq starttls-use-gnutls t)

(require 'smtpmail)
(setq send-mail-function  'smtpmail-send-it
      message-send-mail-function    'smtpmail-send-it
      starttls-use-gnutls t
      smtpmail-starttls-credentials  '(("smtp.office365.com" 587 nil nil))
      smtpmail-auth-credentials (expand-file-name "~/.authinfo.gpg")
      smtpmail-smtp-server  "smtp.office365.com"
      smtpmail-stream-type  'starttls
      smtpmail-smtp-service 587)

\end{minted}

\subsection{Mu4e}
\label{sec:org26b7cb4}

On a mac install mu via brew:

\begin{verbatim}
brew install mu --with-emacs --HEAD
\end{verbatim}

and make sure the path below points to the same \texttt{HEAD} directory!

\begin{verbatim}

(cond
 ((string-equal system-type "windows-nt") ; Microsoft Windows
  (progn
    (message "Microsoft Windows")))
 ((string-equal system-type "darwin") ; Mac OS X
  (progn
    (add-to-list 'load-path "/usr/local/Cellar/mu/HEAD-7d6c30f/share/emacs/site-lisp/mu/mu4e")
    (setq mu4e-mu-binary "/usr/local/bin/mu")
    ))
 ((string-equal system-type "gnu/linux") ; linux
  (progn
    ;;  (add-to-list 'load-path "~/programs/mu/mu4e")
    (add-to-list 'load-path "/usr/share/emacs/site-lisp/mu4e")
    (add-to-list 'load-path "/usr/local/share/emacs/site-lisp/mu4e")   
    ;;         (setq mu4e-mu-binary "/usr/local/bin/mu")
    )))

(message "Loading Mu4e...")

;;  (add-to-list 'load-path "~/programs/mu/mu4e")

;;         (add-to-list 'load-path "/usr/local/share/emacs/site-lisp/mu/mu4e")   

;; the modules
;;(if (file-exists-p mu4e-mu-binary)
;;    (message "Loading Mu4e...")


(if (not (require 'mu4e nil t))
    (message "`mu4e' not found")

  ;;(require 'mu4e)
  (require 'org-mu4e)

  (setq mu4e-maildir (expand-file-name "~/Maildir"))

  (setq mu4e-sent-folder "/office365/sent")
  (setq mu4e-drafts-folder "/drafts")
  (setq mu4e-refile-folder "/office365/Archive")   ;; saved messages
  (setq mu4e-trash-folder "/office365/trash")


  (setq message-kill-buffer-on-exit t)
  (setq mu4e-change-filenames-when-moving t)
  (setq mu4e-confirm-quit nil)
  (setq mail-user-agent 'mu4e-user-agent)
  (setq mu4e-completing-read-function 'ivy-completing-read)

  (setq mu4e-sent-messages-behavior 'sent)

  (setq mu4e-view-show-addresses t)

  (setq mu4e-attachment-dir "~/Downloads")


  (define-key mu4e-headers-mode-map (kbd "C-c c") 'org-mu4e-store-and-capture)
  (define-key mu4e-view-mode-map (kbd "C-c c") 'org-mu4e-store-and-capture)

  (setq mu4e-get-mail-command "offlineimap")

  (setq mu4e-compose-signature
        "Associate Professor Timo Lassmann
Feilman Fellow
Academic Head of Computational Biology, Telethon Kids Institute
Adjunct Associate Professor, Center for Child Health Research
University of Western Australia

Telethon Kids Institute
Northern Entrance, Perth Children's Hospital
15 Hospital Avenue, Nedlands, Western Australia, 6009
PO Box 855, West Perth, Western Australia, 6872 

https://scholar.google.com.au/citations?user=7fZs_tEAAAAJ&hl=en

Visiting Scientist, RIKEN Yokohama Institute, Japan
Division of Genomic Technology,
RIKEN Center for Life Science Technologies,
Yokohama Institute,1-7-22 Suehiro-cho,
Tsurumi-ku, Yokohama, 230-0045 JAPAN")
  )
\end{verbatim}



Spell check 

\begin{verbatim}
(add-hook 'mu4e-compose-mode-hook
          'flyspell-mode)
(add-hook 'message-mode-hook 'turn-on-orgtbl)
(add-hook 'message-mode-hook 'turn-on-orgstruct++)
(add-hook 'mu4e-compose-mode-hook 'turn-off-auto-fill)
\end{verbatim}

\section{TRAMP}
\label{sec:org4f680d2}

\begin{minted}[frame=lines,linenos=true]{common-lisp}
(use-package tramp
  :ensure t
  :config
  (with-eval-after-load 'tramp-cache
    (setq tramp-persistency-file-name "~/.emacs.d/tramp"))
  (setq tramp-default-method "ssh")
  (setq tramp-use-ssh-controlmaster-options nil) 
  (message "tramp-loaded"))
\end{minted}

\section{Autoinsert templates}
\label{sec:orgde82a0c}

Again from Howards Abrams: 
\begin{minted}[frame=lines,linenos=true]{common-lisp}
(use-package autoinsert
  :ensure t
  :init
  (setq auto-insert-directory (tl/emacs-subdirectory "templates/"))
  ;; Don't want to be prompted before insertion:
  (setq auto-insert-query nil)

  (add-hook 'find-file-hook 'auto-insert)
  (auto-insert-mode 1))

\end{minted}

Use yes snippet for templates. 

\begin{minted}[frame=lines,linenos=true]{common-lisp}
(defun tl/autoinsert-yas-expand()
  "Replace text in yasnippet template."
  (yas-expand-snippet (buffer-string) (point-min) (point-max)))
\end{minted}

Set templates

\begin{minted}[frame=lines,linenos=true]{common-lisp}
(use-package autoinsert 
  :config
  (define-auto-insert "\\.org$" ["default-orgmode.org" tl/autoinsert-yas-expand]))
\end{minted}

\section{Eshell}
\label{sec:org691d936}

Set up environment.
\begin{minted}[frame=lines,linenos=true]{common-lisp}

(setenv "LD_LIBRARY_PATH" "/usr/local/lib")
;;(setenv "PATH"
;;        (concat
;;         "/usr/local/bin:/usr/local/sbin:"
;;         (getenv "PATH")))

\end{minted}

\begin{minted}[frame=lines,linenos=true]{common-lisp}
(use-package eshell
  :init
  (setq ;; eshell-buffer-shorthand t ...  Can't see Bug#19391
   eshell-scroll-to-bottom-on-input 'all
   eshell-error-if-no-glob t
   eshell-hist-ignoredups t
   eshell-save-history-on-exit t
   eshell-prefer-lisp-functions nil
   eshell-destroy-buffer-when-process-dies t))
\end{minted}

\begin{minted}[frame=lines,linenos=true]{common-lisp}
(use-package eshell
  :init
  (add-hook 'eshell-mode-hook
            (lambda ()
              (add-to-list 'eshell-visual-commands "ssh")
              (add-to-list 'eshell-visual-commands "tail")
              (add-to-list 'eshell-visual-commands "top"))))
\end{minted}

Alias
\begin{minted}[frame=lines,linenos=true]{common-lisp}
(add-hook 'eshell-mode-hook (lambda ()
                              (eshell/alias "e" "find-file $1")
                              (eshell/alias "ff" "find-file $1")
                              (eshell/alias "emacs" "find-file $1")
                              (eshell/alias "ee" "find-file-other-window $1")

                              (eshell/alias "gd" "magit-diff-unstaged")
                              (eshell/alias "gds" "magit-diff-staged")
                              (eshell/alias "d" "dired $1")
                              (eshell/alias "val" "valgrind --leak-check=yes --show-leak-kinds=all --exit-on-first-error=yes --error-exitcode=1 $*")

                              ;; The 'ls' executable requires the Gnu version on the Mac
                              (let ((ls (if (file-exists-p "/usr/local/bin/gls")
                                            "/usr/local/bin/gls"
                                          "/bin/ls")))
                                (eshell/alias "ll" (concat ls " -AlohG --color=always")))))
\end{minted}

\section{PDF tools}
\label{sec:org1852924}

\begin{minted}[frame=lines,linenos=true]{common-lisp}
(use-package pdf-tools
  :config
  ;; initialise
  (pdf-tools-install)
  ;; open pdfs scaled to fit page
  (setq-default pdf-view-display-size 'fit-page)
  ;; automatically annotate highlights
  (setq pdf-annot-activate-created-annotations t)
  ;; use normal isearch
  (define-key pdf-view-mode-map (kbd "C-s") 'isearch-forward))
\end{minted}

\section{End}
\label{sec:org925755f}


Run client

\begin{minted}[frame=lines,linenos=true]{common-lisp}


(if (daemonp)
    (add-hook 'after-make-frame-functions
              '(lambda (f)
                 (with-selected-frame f
                   (when (window-system f) (require 'init-client) ))))
  (require 'init-client) )



\end{minted}

\begin{minted}[frame=lines,linenos=true]{common-lisp}
(require 'init-local nil t)
\end{minted}






\subsection{Fill Mode}
\label{sec:orgf516965}
Automatically wrapping when you get to the end of a line (or the fill-region):


\begin{verbatim}




(use-package fill
  :bind (("C-c T f" . auto-fill-mode)
         ("C-c T t" . toggle-truncate-lines))
  :init (add-hook 'org-mode-hook 'turn-on-auto-fill)
  :diminish auto-fill-mode)

\end{verbatim}


End. 
\end{document}
